\documentclass{article}
\usepackage{graphicx}
\usepackage{fullpage}
\usepackage{hyperref}
\usepackage{amsmath}
\usepackage{amssymb}
\usepackage{draftwatermark}

\SetWatermarkText{DRAFT}
\SetWatermarkScale{3}
\SetWatermarkLightness{0.5}

\begin{document}

\title{EP 501 Homework 5:  Ordinary Differential Equations}

\maketitle

~\\
\textbf{Instructions:}  
\begin{itemize}
  \item Submit all MATLAB or Python source code and results via Canvas.  Please zip all contents of your solution into single file and then submit in a single zip file.    
  \item Discussing the assignment with others is fine, but you must not copy anyone's code.  
  \item I must be able to run your code and produce all results by executing a single top-level Matlab script, e.g. \textsf{assignment6.(m,py)} or similar.  
  \item You may use any of the example codes from our course repositories:  \url{https://github.com/Zettergren-Courses/EP501_python} and \url{https://github.com/Zettergren-Courses/EP501_matlab}.
  \item Do not copy verbatim any other codes (i.e. any source codes other than from our course repository).  You may use other examples as a reference but you must write you own programs (except for those I give you).  

\end{itemize}
~\\~\\~\\
\textbf{Purpose of this assignment:}  
\begin{itemize}
  \item Solve ODEs numerically and interpret the results.  
  \item Develop good coding and documentation practices, such that your programs are easily understood by others.  
  \item Exercise good judgement in numerical problem setup.
  \item Demonstrate higher reasoning to synthesize a problem and devise a basic algorithm to solve it.  
\end{itemize}

\pagebreak

\begin{enumerate}
  \item  The electrostatic potential in a region $-a \le x \le a$ with continuously varying dielectric material having dielectric function $\epsilon(x)$, is governed by the equation:  
  \begin{equation}
    \epsilon \frac{d^2 \Phi}{d x^2} + \frac{d \epsilon} {d x} \frac{d \Phi}{d x} = 0 
  \end{equation}
  Suppose the dielectric function takes the form:
  \begin{equation}
%    \epsilon(x) = \epsilon_0 \left( 10 - \frac{10}{2} \tanh \left( \frac{x-x'}{\ell}\right) + \frac{10}{2} \tanh \left( \frac{x-x''}{\ell}\right) \right)
    \epsilon(x) = \epsilon_0 \left( 10 \tanh \left( \frac{x-x'}{\ell}\right) - 10 \tanh \left( \frac{x-x''}{\ell}\right) \right)
  \end{equation} 
  with parameters:
  \begin{eqnarray}
  a &=& 0.01 \qquad (\mathrm{m}) \nonumber \\
  \ell &=& \frac{a}{5} \qquad (\mathrm{m}) \nonumber \\
  x' &=& -\frac{9a}{10} \qquad (\mathrm{m}) \nonumber \\
  x'' &=& \frac{9a}{10} \qquad (\mathrm{m}) \nonumber  
  \end{eqnarray}  
  The boundary conditions for this system are given by:
  \begin{eqnarray}
  \frac{d \Phi}{d x}(-a) &=& 1000 \qquad (\mathrm{V/m}) \nonumber \\
  \Phi(a) &=& 100 \qquad (\mathrm{V}) \nonumber
  \end{eqnarray}     
  \begin{itemize}
    \item[(a)]  Plot the dielectric function and note that it varies rapidly near the edges of the domain of interest.
    \item[(b)]  Develop a system of finite difference equations for this system based on second order accurate centered differences (numerical differences may also be used for dielectric function derivatives).  Include this system in your homework submission.
    \item[(c)]  Develop two finite difference equations for the boundary conditions of this system.  Use a first-order forward difference at the $x=-a$ boundary.  Include these equations with your submission.  
    \item[(d)]  Solve your system of equations using the MATLAB ``$\backslash$'' operator.
    \item[(e)]  Since the dielectric function varies rapidly at the boundary, this is a problem where a second order (forward) difference may be useful (see course notes for formula).  Reformulate your matrix system to include this for the $x=-a$ boundary and solve the system numerically for 20 grids points.  Plot the result and compare it against the solution with a first order forward difference.  
  \end{itemize}
  \item  Consider a charged particle immersed in a magnetic field $\mathbf{B} = B \hat{\mathbf{e}}_z$.  If the field is uniform and the particle is given an initial velocity the particle will oscillate in a perfect circular motion as shown in class an in the example system of ODEs in the course repository, which solves the system of equations:
    \begin{equation}
      m \frac{d \mathbf{v}}{dt} = q \mathbf{v} \times \mathbf{B}
    \end{equation}  
    ($\mathbf{v}=v_x \hat{\mathbf{e}}_x + v_y \hat{\mathbf{e}}_y$) using a two-step, second order accurate Runge-Kutta method (RK2).  Assume that the $v_z$ is constant with time.  Use the parameters:  
    \begin{eqnarray}
      m &=& 1.67 \times 10^{-27} \quad \mathrm{[kg]} \\
      q &=& 1.6 \times 10^{-19} \quad \mathrm{[C]} \\
      B &=& 50000 \quad \mathrm{[nT]} \\
      v_x(t=0) &=& 1 \quad \mathrm{[km/s]} \\
      v_y(t=0) &=& 1 \quad \mathrm{[km/s]} 
    \end{eqnarray}
  \begin{itemize}
    \item[(a)]  Solve this system with the RK4 algorithm and plot the velocity and position as a function of time.  Compare this against the RK2 solution in the repository and show they are roughly equal.  Use 100 time steps per particle oscillation period $T=\frac{2 \pi m}{q B}$.  
    \item[(b)]  Show that the RK4 solution is better in the sense that it can solve the problem accurately with fewer time steps.  
    \item[(c)]  Suppose the magnetic field is changed to vary linearly in the $y$-direction:
    \begin{equation}
      \mathbf{B}(y) = B \left( 1+\frac{1}{2}y \right) \hat{\mathbf{e}}_z
    \end{equation}
    Use your RK4 solver to find the velocity.  Plot the velocities and path of the particle in the $x$-$y$ plane for a least five periods of oscillation.  HINT:  The particle should execute trochoidal motion.  
  \end{itemize} 
  \item \emph{Efficiently} solve the system of equations:
  \begin{eqnarray}
    \frac{d y_1}{d t} &=& 998 y_1 +1998 y_2 \\
    \frac{d y_2}{d t} &=& -999 y_1 -1999 y_2
  \end{eqnarray}
    for $0 \le t \le 0.5$ if $y_1(0)=y_2(0)=1$.  It is only necessary to resolve the longest time scale present in this system and a first-order accurate solution is acceptable.  You may use any codes that we have developed in the course repository.  Points will be deducted for solutions using a method that requires an overly large number of time steps.  

\end{enumerate}

\end{document}
