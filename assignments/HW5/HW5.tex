\documentclass{article}
\usepackage{graphicx}
\usepackage{fullpage}
\usepackage{hyperref}
\usepackage{amsmath}
\usepackage{amssymb}
\usepackage{draftwatermark}

\SetWatermarkText{DRAFT}
\SetWatermarkScale{3}
\SetWatermarkLightness{0.5}

\begin{document}

\title{EP 501 Homework 5:  Differentiation and Integration}

\maketitle

~\\
\textbf{Instructions:}  
\begin{itemize}
  \item Submit all MATLAB or Python source code and results via Canvas.  Please zip all contents of your solution into single file and then submit in a single zip file.    
  \item Discussing the assignment with others is fine, but you must not copy anyone's code.  
  \item Please arrange your code so that I can run a single script and produce all results by executing, e.g. \textsf{assignment1.(m,py)} or similar.  
  \item You may use any of the example codes from our course repositories:  \url{https://github.com/Zettergren-Courses/EP501_python} and \url{https://github.com/Zettergren-Courses/EP501_matlab}.
  \item Do not copy verbatim any other codes (i.e. any source codes other than from our course repository).  You may use other examples as a reference but you must write you own programs (except for those I give you).  

\end{itemize}
~\\~\\~\\
\textbf{Purpose of this assignment:}  
\begin{itemize}
  \item Use numerical differentiation to solve complex problems.  
  \item Develop good coding and documentation practices, such that your programs are easily understood by others.  
  \item Exercise good judgement in numerical problem setup.
  \item Demonstrate higher reasoning to synthesize a problem and devise a basic set of algorithms to solve it.  
\end{itemize}

\pagebreak

\begin{enumerate}
  \item  Numerical Vector Derivatives (curl):  
  \begin{itemize}  
    \item[(a)] Plot the two components of the vector magnetic field defined by the piecewise function:
    \begin{equation}
      \mathbf{B}(x,y)= \left\{
      \begin{array}{c}
      \frac{\mu_0 I}{2 \pi a^2} \sqrt{x^2+y^2} \left( - \frac{y}{\sqrt{x^2+y^2}}\hat{\mathbf{e}}_x + \frac{x}{\sqrt{x^2+y^2}} \hat{\mathbf{e}}_y \right) \qquad \left( \sqrt{x^2+y^2} < a \right)
      \\
      \frac{\mu_0 I}{2 \pi \sqrt{x^2+y^2}} \left( - \frac{y}{\sqrt{x^2+y^2}} \hat{\mathbf{e}}_x + \frac{x}{\sqrt{x^2+y^2}} \hat{\mathbf{e}}_y \right) \qquad \left( \sqrt{x^2+y^2} \ge a \right)
      \end{array}
      \right.      
    \end{equation}
    Assume the parameters in this equation have the numerical values:
    \begin{eqnarray}
      I &=& 10 \qquad \left( \mathrm{A} \right) \nonumber \\
      \mu_0 &=& 4 \pi \times 10^{-7} \qquad (\mathrm{H/m}) \nonumber \\
      a &=& 0.005 \qquad (\mathrm{m}) \nonumber 
    \end{eqnarray}
    Use an image plot (e.g. \texttt{pcolor} and \texttt{shading flat} in MATLAB or \texttt{matplotlib.pyplot.pcolor} in Python) for each magnetic field component ($B_x,B_y$) and have your plot show the region $-3 a \le x \le 3 a, -3a \le y \le 3a$.  Make sure you add a colorbar and axis labels to your plot.  You will need to define a range and resolution in $x$ and $y$, and create a meshgrid from that.  Be sure to use a resolution fine enough to resolve important variations in this function.  
    \item[(b)]  Make a quiver plot of the magnetic field $\mathbf{B}$; add labels, etc.
    \item[(c)]  Compute the numerical curl of $\mathbf{B}$, i.e. $\nabla \times \mathbf{B}$.  Use centered differences on the interior grid points and first-order derivatives on the edges.  Plot your result using \texttt{imagesc}, or \texttt{pcolor}.  
    \item[(d)]  Compute $\nabla \times \mathbf{B}$ analytically (viz. by hand).  Plot the  alongside your numerical approximation and demonstrate that they are suitably similar. 
  \end{itemize} 
  \item Numerical Vector Derivatives (gradient and laplacian):
  \begin{itemize}
    \item[(a)]  Compute and plot the scalar field:
    \begin{equation}
      \Phi(x,y,z)= \left\{
      \begin{array}{c}
      \frac{Q}{4 \pi \epsilon_0 a} - \frac{Q}{8 \pi \epsilon_0 a^3} \left(x^2+y^2+z^2 - a^2 \right)
      \qquad \left( \sqrt{x^2+y^2+z^2} < a \right)
      \\
      \frac{Q}{4 \pi \epsilon_0 \sqrt{x^2+y^2+z^2}} \qquad \left( \sqrt{x^2+y^2+z^2} \ge a \right)
      \end{array}
      \right.      
    \end{equation}
    Use the parameters:
    \begin{eqnarray}
      Q &=& 1  \qquad \left( \mathrm{C} \right) \nonumber \\
      a &=& 1 \qquad (\mathrm{m}) \nonumber \\
      \epsilon_0 &=& 8.854 \times 10^{-12} \qquad (\mathrm{F/m}) \nonumber 
    \end{eqnarray}    
    and plot this function in the region $-3a \le x \le 3a, -3a \le y \le 3a$ in the $z=0$ plane. Be sure to use a resolution fine enough to resolve variations in this function (aside from those associated with the singularity).
    \item[(b)]  Write a function to numerically compute the Laplacian of a scalar field, i.e. $\nabla^2 \Phi$.  Plot your result with appropriate labels and colorbars.  
    \item[(c)]  Compute an analytical laplacian (viz. differentiate by hand), plot the results alongside your numerical calculation, and demonstrate that your numerical laplacian is suitably accurate.  
  \end{itemize}
  \item  Integration in Multiple Dimensions.  
  \begin{itemize}
    \item[(a)] Numerically compute the electrostatic energy in the region $R \equiv -3a \le x \le 3a, -3a \le y \le 3a, -3a \le z \le 3a$, defined by the integral:  
    \begin{equation}
      W_E = - \frac{1}{2} \iiint_R \left( \epsilon_0 \nabla^2 \Phi \right) \Phi ~dx dy dz
    \end{equation}
    using an iterated trapezoidal method (sweeps of single dimensional integrations) or multi-dimensional trapezoidal method program that you write.  
  \end{itemize}
  \item Line Integration:  
  \begin{itemize}
    \item[(a)]  Compute and plot the parametric path
    \begin{equation}
       \mathbf{r}(\phi) \equiv x(\phi)\hat{\mathbf{e}}_x + y(\phi)\hat{\mathbf{e}}_y = r_0 \cos \phi ~ \hat{\mathbf{e}}_x + r_0 \sin \phi ~ \hat{\mathbf{e}}_y \qquad (0 \le \phi \le 2 \pi)
    \end{equation}    
    in the $x,y$ plane on the same axis as your magnetic field components from problem 1 (create a new figure which plots the path on top of a \texttt{pcolor} plot of each component).  Take $r_0 = 2a$.  You will need to define a grid in $\phi$ to do this.
    \item[(b)]  Plot the two components of the magnetic field $\mathbf{B}(x(\phi),y(\phi))$ at the $x,y$ points along $\mathbf{r}$ and visually compare against your image plots of the magnetic field and path to verify.  
    \item[(c)]  Numerically compute the tangent vector to the path $\mathbf{r}$ by performing the derivative:  
    \begin{equation}
      \frac{d \mathbf{r}}{d \phi} =  \frac{d x}{d \phi} \hat{\mathbf{e}}_x + \frac{d y}{d \phi} \hat{\mathbf{e}}_y
    \end{equation}
    Compare your numerical results against the analytical derivative (e.g. plot the two) and refine your grid in $\phi$ (if necessary) such that you get visually acceptable results - i.e. such that the path appears circular.  
    \item[(d)]  Numerically compute the auxiliary magnetic field integrated around the path $\mathbf{r}$, i.e.:
    \begin{equation}
       I = \int_{\mathbf{r}} \frac{\mathbf{B}}{\mu_0} \cdot d \boldsymbol{\ell} 
    \end{equation}
    where the differential path length is given by:
    \begin{equation}
      d \boldsymbol{\ell} = \frac{d \mathbf{r}}{d \phi} d \phi 
    \end{equation}   
  \end{itemize}
\end{enumerate}

\end{document}
